\chapter*{Conclusion}
\addcontentsline{toc}{chapter}{Conclusion}

À partir d'un corpus de 696 pièces textiles andines, nous nous sommes attelés dans ce mémoire à examiner les mobilités et les imitations techniques et iconographiques dans les Andes grâce aux outils numériques. Ce mémoire se fondait sur l'hypothèse de l'existence d'échanges de textiles dans la zone andine, accompagnés de phénomènes de ré-interprétations techniques et iconographiques. \\

Dans la première partie, nous nous sommes appuyés sur la littérature des textiles andins afin de comprendre la diversité des objets sur lesquels nous travaillons. Puis nous avons présenté les différentes modalités de circulations dont nous avons connaissance, ainsi que les influences techniques et iconographiques présentes dans l'état de l'art. Ce travail de recension sur plusieurs millénaires nous a permis de présenter quelques cas d'étude de la base de données qui attestent de ces circulations et justifient le recours aux outils numériques afin d'élargir ces observations à l'ensemble des pièces textiles du corpus. %Concordance des temps ?

Dans la seconde partie, nous sommes revenus sur les informations géographiques disponibles dans la base de données afin de les visualiser et d'en tirer des analyses. Ce travail a nécessité un lourd processus de pré-traitement des données, notamment lié à la complexité de la chronologie andine. L'analyse des métadonnées associées aux textiles était également l'occasion de comprendre les choix réalisés par les chercheurs et chercheuses à l'origine de la base de données. Ils se sont avérés déterminants pour notre traitement de données. Après avoir effectué ce travail critique et surpassé les écueils du géocodage, nous avons analysé les foyers de productions textiles dans une approche diachronique, confirmant la distinction entre textiles contemporains des hautes-terres et textiles archéologiques côtiers. Récupérer des informations géolocalisées nous a également permis de mettre en évidence certains parcours d'objets. En retraçant grossièrement ces potentiels trajets, nous avons attesté d'une intense circulation des textiles entre civilisations côtières et confirmé des trajets présents dans la littérature. 

Au cours de la troisième et dernière partie, nous avons approfondi les influences entre les textiles à partir de leurs photographies. Dans un premier temps, nous avons appliqué une classification automatique des armures textiles, reposant sur leurs métadonnées. Nous avons prouvé que les réseaux de neurones convolutifs les plus récents sont en mesure de réaliser une classification probante des armures textiles andines, outil utile dans un cadre patrimonial. En outre, en adaptant cette même technique, nous avons montré que les méthodes de l'intelligence artificielle sont une piste convaincante pour trouver des ré-interprétations iconographiques dans les textiles. Grâce aux \textit{clustering} des \textit{features} d'images, nous avons distingué l'iconographie côtière de celle des hautes-terres mais également des cas d'imitations et de ré-interprétations iconographiques et techniques. L'iconographie des hautes-terres semble s'être largement répandue sur la côte, notamment à la période Intermédiaire Tardive au cours de laquelle on détecte à la fois des circulations et des imitations. Nous avons également relevé que les rares textiles coloniaux de la base de données s'inscrivent du côté des textiles républicains, sans héritage particulier des pratiques textiles pré-hispaniques. Ce chapitre ouvre la voie à un approfondissement de cette méthode pour saisir de manière plus systématiques les transferts de savoir-faire dans la zone andine. \\

Ce mémoire dans son ensemble participe à la remise en question récente de l'immuabilité de l'~\og~andinité~\fg, encore très présente dans les études andinistes, à la fois en archéologie, en histoire et en anthropologie. L'analyse de notre base de données illustre en effet la variété des pratiques textiles dans les Andes et les multiples iconographies présentes sur le territoire au cours du temps. Ainsi, les civilisations citées par Ciriaco lors de notre entretien donnent à voir l'héritage complexe des artisan\inclusives{ne} contemporain\inclusives{ne} des Andes. 

Ce mémoire s'inscrit également dans la continuité des études numériques du patrimoine textile. Comme le souligne Jorge Sebastián Lozano et son équipe, \og le patrimoine textile est un des nombreux \textquotedblleft parents pauvres\textquotedblright \:dans la grande famille du patrimoine culturel \fg\footcite[p.~76]{sebastianlozanoCatalogosMuseoGran2020}, notamment par sa faible numérisation, et mériterait d'être plus systématique analysé. Notre travail illustre cette nécessité de numérisation des collections textiles puisque nous détenons des outils qui bénéficient de l'accès à des pièces textiles appartenant à des collections muséales situées à des milliers de kilomètres. Néanmoins, ce travail de numérisation doit s'accompagner d'une bonne qualité d'image, ce qui n'était pas le cas pour tous les textiles de la base de données et a compliqué la tâche de vérification des techniques, notamment des relevés textiles.\\

\noindent D'autres limites sont apparues au fil de cette étude, à la hauteur de la complexité de notre objet : 
\begin{itemize}
	\item Du point de vue du traitement géographique, nous aurions pu explorer les autres données qualitatives présentes dans la base de données sous forme de cartographies thématiques. Ainsi, il aurait pu être intéressant de cartographier les matériaux composant les textiles, puisque nous savons qu'ils sont une des traces des circulations entre hautes-terres et côte. 
	\item La détection des techniques mériterait également d'être approfondie. Il serait intéressant de tester ces méthodes sur un corpus élargi, pour tester sa robustesse. En outre, un travail de meilleure compréhension du processus interne des algorithmes est nécessaire. Nous avons essayé de visualiser les différentes couches de \textit{features} du réseau afin de comprendre la manière dont se fait la classification mais les résultats n'étaient pas probants. Toutefois, cette perspective devrait être approfondie, notamment pour pallier le poids de la prise de vue dans la classification non-supervisée. Cet écueil pourrait également être surpassé en ayant recours à un pré-traitement des images pour ne fournir que des détails de textiles ou bien nous aurions pu tenter de classifier les textiles (et non pas les images) en fournissant des \textit{batchs} d'images, c'est-à-dire plusieurs images par textile.
\end{itemize}

Néanmoins, les résultats obtenus sont prometteurs et permettent de prolonger ce travail de compréhension des textiles andins et à travers eux de connaissance des populations qui les produisaient\footcite[p.~109]{conklinStructureMeaningAndean1997}. 


%Critique : source pas forcément fiable (Victoria & Albert Museum pas fiable d'après Sophie Desrosiers) --> dépendant du travail muséal en amont = qualité des photos, fiabilité des analyses etc.
%NB : un tel travail pourrait également être élargi à d'autres supports --> poterie ou architecture pour lesquels on a plus de cas (comme le fait Desrosiers \footcite[p.~7]{desrosiersHighlandComplementaryWarpWeaving2014} = transmédialité de l'art (Schultz)


