\chapter*{Glossaire des termes textiles\footnote{Ce glossaire a été réalisé à partir des sources suivantes : \begin{itemize} \item A. Castres et T. Gaumy, \textit{La fabrique de l'habit: artisans, techniques et production du vêtement (XVe-XVIIIe siècle)}, Paris, École des chartes, 2020. \item Centre International d'Étude des Textiles Anciens, \textit{Vocabulaire technique français}, Lyon, CIETA, 2020. \item I. Emery, \textit{The primary structures of fabrics: an illustrated classification}, Londres, Paris, Thames and Hudson, The Textile Museum, 1995[1966]. \item A. Leroi-Gourhan, \textit{L'Homme et la Matière : évolution et techniques}, Paris, Albin Michel, 1943. \end{itemize} }}
\addcontentsline{toc}{chapter}{Glossaire}

\begin{itemize}
	\item \textbf{Tissage} : Assemblage de deux nappes d'éléments perpendiculaires, l'une longitudinale nommée \textbf{chaîne} et l'autre transversale nommée \textbf{trame}. 
\end{itemize}

\vspace{10pt}

 \begin{itemize}
 	\item \textbf{Armure textile ou structure textiles} : Manière dont les fils sont entremêlés.
	\item \textbf{Broderie} : Technique d'ornementation qui consiste à ajouter à la surface d'un support un décor généralement exécuté à l'aiguille avec des fils et divers éléments.
	\item \textbf{Chaîne ou trame supplémentaire} : Groupe d'éléments supplémentaires parcourant le tissu dans le sens de la trame ou de la chaîne et répondant à une fonction distincte : création de motifs, effet de texture, renforcement du tissage...
	\item \textbf{Double-étoffe} : Tissu composé de deux tissages distincts l'un au-dessus de l'autre, échangeant fréquemment leurs positions respectives dans le tissu.
	\item \textbf{Flotté} : Tout fil de trame ou de chaîne qui passe au dessus de plus d'un fil de l'élément opposé.
	\item \textbf{Lisière} : Bord d'un tissu.
	\item \textbf{Réseau} : Tissu composé d'un fil unique qui s'enlace ou se noue sur lui-même.
	\item \textbf{Tissage équilibré ou toile} : tissage composé d'une seule trame et d'une seule chaîne, donc chaque fil passe successivement au-dessus puis au-dessous des éléments opposés, sans flottés.
	\item \textbf{Tissage face chaîne ou dominante chaîne} : Tissage au sein duquel le nombre et la densité des fils de chaîne font qu'ils cachent entièrement les fils de toile.
	\item \textbf{Tissage face trame ou dominante trame ou tapisserie} : Tissage au sein duquel le nombre et la densité des fils de trame font qu'ils cachent entièrement les fils de chaîne.
\end{itemize}