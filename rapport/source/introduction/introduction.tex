\chapter*{Introduction}
\addcontentsline{toc}{chapter}{Introduction}
\setlength{\columnsep}{3em}

\footnotelayout{m} \begin{paracol}{2}
 \noindent  \og Parce qu'il y a beaucoup de gens dans mon pays qui ne savent plus ce qu'est Paracas, ce qu'est Huari, et Chancay et Nazca, tu vois ? Presque plus personne ne sait maintenant. Donc nous som- mes, je suis pour la diffusion, et je suis pour l'enseigner, tu vois ? Pour qu'ils puissent apprendre, pour qu'ils puissent me connaître, connaître mon travail ou le travail que nous faisons depuis les ancêtres tu vois ? [...] C'est qu'au Pérou, il y a beaucoup, beaucoup de lieux qui ont fait [des textiles] comme Huari, comme Nazca, comme Chancay et comme Paracas. [...] Alors, c'est pour ça que les gens et nous travaillons. Parce que nous ne voulons pas perdre l'essence que nous a laissée l'héritage, nos ancêtres, tu vois \footnote{Entretien avec Ciriaco, 13 mars 2022, Ayacucho (Pérou).} ? \fg \\
    
    \switchcolumn
    
    \textquotedblleft \textit{Porque hay mucha gente en mi país ya no saben qué cosa es Paracas, qué cosa es este Huari ?`no? y Chancay y este Nazca?`no? Casi nadie saben ya no. Entonces nosotros estamos esta, yo estoy esta para difundir y yo estoy para enseñarla ?`no?, para que puedan aprender, para que pueden conocer a mi, a mi, a mi trabajo o el trabajo que hacemos desde los ancestros ?`no? [...] Es que en Perú, hay muchos, muchos lugares que, que hubo [tejidos] pues como Huari, como Nazca, como Chancay y como Paracas. Entonces hay muchas leyendas de tejido. Muchas leyendas. [...] Entonces, hacia eso la gente o nosotros trabajamos ?`no? Porque no quisiéramos perder esa esencia que nos ha dejado la herencia, el antepasado ?`no?}\textquotedblright\\
\end{paracol}


Ciriaco, tisserand à Ayacucho dans les Andes centrales péruviennes, ancre son travail d'artisan textile dans la continuité d'une longue tradition. Lors de cette discussion, il présente le textile comme vecteur de l'héritage préhispanique en l'inscrivant dans la lignée de civilisations disparues : Paracas, Huari, Chancay ou Nazca. Le textile qu'il produit est commercialisé comme artisanat péruvien mais également présenté comme héritier de ces cultures préhispaniques dont Ciriaco se revendique descendant\footnote{Les entretiens ethnographiques ne peuvent être compris hors de la manière dont les enquêté\inclusives{e} souhaitent être perçus. Quelques minutes avant d'aborder cette thématique, Ciriaco me demande si je suis toujours en train d'enregistrer, il a conscience que le discours qu'il me tient restera gravé et présente possiblement ce qu'il pense que j'attends de cette discussion.}.
	
Ce discours de continuité de la pratique textile est fortement développé au Pérou, et plus largement dans les Andes, car une multitude de textiles a été retrouvée lors des nombreuses fouilles archéologiques qui ont eu lieu depuis le milieu du \siecle{xx}. Si ces découvertes sont possibles c'est notamment grâce au climat désertique le long de la côte péruvienne qui offre des conditions de conservation idéale pour les fibres animales et végétales qui composent les pièces textiles\footnote{\cite[p.~263]{desrosiersTechniquesTissageOntelles2010}}. À l'inverse, les civilisations des hautes-terres de la Cordillère des Andes (ou \textit{sierra}) ont été productrices de textiles, mais le climat humide et froid en altitude n'a pas permis leur conservation --- à part quelques échantillons retrouvés sur des momies découvertes au-dessus de 5000 mètres d'altitude et dont les vêtements ont été conservés par la congélation\footnote{\cite{abalderussoArteTextilIncaico2010}}--- .
\noindent Ces textiles perdurent sur la longue durée, puisqu'ils composent un témoignage de cultures matérielles s'étendant sur plus d'un millénaire, et sont encore inscrits dans l'imaginaire des tisserand\inclusives{e} contemporain\inclusives{e} comme l'illustre le discours de Ciriaco. \\


\section*{État de l'art}

\subsubsection*{La foisonnante étude des textiles andins}

L'abondance des textiles andins, leur complexité et leur ancienneté génèrent, comme le souligne Annabel Vallard, \og une passion internationale et une bibliographie abondante \fg\footcite[p.~1]{vallardSilvermanGailWoven2009}.

Dès les années 1930, les études américanistes se concentrent sur les découvertes archéologiques préhispaniques, y compris sur le textile. Lila O'Neale et Alfred Kroeber proposent ainsi, dans un recueil en majeure partie composé de planches photographiques, des pistes d'analyse des textiles archéologiques découverts sur la côte péruvienne au début du \siecle{xx}\footcite{onealeTextilePeriodsAncient1930}. 
Dès ces premiers travaux de recensions des fouilles, les archéologues développent une volonté classificatoire, notamment selon les cultures pré-hispaniques. En France, Raoul D'Harcourt, précurseur de l'ethnographie andiniste française, propose une analyse des textiles archéologiques du Pérou et de leurs techniques en 1934, cette analyse reste une référence dans le domaine\footcite{harcourtTextilesAnciensPerou2008}. Il s'applique à montrer la complexité des techniques utilisées par les populations préhispaniques et en propose une typologie. Une grande partie de l'analyse du textile, y compris dans le champ andiniste, consiste alors à penser la classification des artefacts. André Leroi-Gourhan, en 1943, dans sa classification générale des techniques, analyse les moyens de production textiles, y compris ceux utilisés dans les Andes. Il classe les textiles dans la catégorie des \og solides souples \fg\:composés d'élément ayant une \og flexibilité permanente qui permet de les assembler par intrication mutuelle \fg \footcite[p.~243]{leroi-gourhanHommeMatiereEvolution1943}. 
Sophie Desrosiers, spécialiste française des textiles andins, se montre critique face à cette classification. Elle souligne que Leroi-Gourhan se contredit dans son étude du textile puisqu'il promeut une classification des techniques à partir des matières premières, mais, pour le textile, il classifie les pièces selon l'assemblage des fibres et non leur composition. En 1966, Irene Emery propose une classification générale des techniques textiles selon l'entrecroisement des fils, accompagnée de photographies et intitulée \textit{The Primary Structure of Fabrics}\footcite{emeryPrimaryStructuresFabrics1995}. Cet ouvrage devient la référence en matière de classification textile et est encore aujourd'hui largement utilisé. Elle prend d'ailleurs des exemples d'armures textiles à partir de pièces archéologiques découvertes au Pérou. Nous pouvons aussi nous référer à l'ouvrage d'Annemarie Seiler-Baldinger pour une classification généraliste du textile selon son mode de production\footnote{\cite{seiler-baldingerTextilesClassificationTechniques1995}.\\Pour le débat sur les différentes proposition de classification textile, voir : \cite{balfetOuSontClassifications1988}}. Ces différentes approches classificatoires, depuis le début du \siecle{xx}, ne sont pas standardisées puisqu'elles ne reposent pas sur les mêmes fondements théoriques. Or, face à la massification des données, quel est le meilleur compromis pour organiser les textiles entre eux sans perdre d'informations ? 

Dans les années 1980, on assiste à une reviviscence de l'analyse des textiles andins. L'anthropologie s'est longtemps intéressée au textile en tant qu'objet et, surtout, comme un objet appartenant au passé, produit par des populations disparues. Toutefois, au début du \siecle{xx}, certaines de ces techniques textiles étaient toujours existantes dans les communautés indigènes, notamment dans les hautes-terres andines. À partir de la fin du \siecle{xx}, les chercheurs et chercheuses s'intéressent donc moins aux textiles archéologiques et à leurs structures qu'aux textiles ethnographiques en eux-mêmes et comme outils de compréhension des textiles archéologiques. Par exemple, Sophie Desrosiers propose la reconstitution d'une ceinture dont le tissage est présenté dans une chronique de Fray Martín de Murúa  au \siecle{xvii}, \textit{Historia del origen y genealogía de los Reyes Incas del Perú} et dont elle a retracé une partie de l'existence jusqu'à l'époque contemporaine\footcite{desrosiersExperienceTechnologieReconstruction1985}. Retracer les évolutions des techniques est une des méthodes proposées par cette même chercheuse pour comprendre les textiles des hautes-terres dont nous avons peu de preuves archéologiques. Puisqu'il est en effet attesté que les textiles andins ont circulé, l'analyse des artefacts découverts sur la côte permettrait de comprendre ce qu'étaient ces textiles. Deux théories principales s'opposent sur cette question. D'une part, les partisans de l'influence principale des hautes-terres sur les régions côtières\footcite{desrosiersHighlandComplementaryWarpWeaving2014}, d'autre part, les tenants d'une influence principale inverse, depuis la côte vers l'Est \footcite{arnoldCienciaTejerAndes2019}. De nouvelles méthodes pourraient-elles nous permettre d'observer les influences mutuelles entre ces deux zones géographiques ? \\

L'histoire est également un domaine fertile pour comprendre l'inscription du textile dans les sociétés andines. 
À la période moderne, les colons espagnols qu'ils soient \textit{conquistadores} ou religieux importent l'écriture et rédigent des témoignages précieux sur les pratiques textiles incaïques. Une partie de ces textes sont des chroniques ; des ouvrages rédigés à la période coloniale par des colons ou par des métis, qui contiennent des descriptions de la colonisation, du mode de vie des habitants colonisés, de la géographie ou de la faune et de la flore locale, ainsi que des récits historiques rapportés par les populations locales. Dès les premières chroniques, la présence du textile est soulignée et largement décrite par les colons et métis. Ces récits historiques sont une source de compréhension des textiles produits dans les hautes-terres dans l'empire Inca et à la période coloniale. John Murra, anthropologue étatsunien de renom, spécialiste de la période inca, développe ainsi une analyse des pratiques textiles incaïques documentée essentiellement à partir des chroniques\footcite{murraClothItsFunctions1962}.

L'influence coloniale européenne sur les pratiques textiles andines a d'abord été étudiée à partir de la thématique des \textit{obrajes}. Quelques auteurs proposent des analyses sur l'introduction de nouvelles races ovines et du métier à pédale au sein des \textit{obrajes}, manufactures textiles destinées à la production de tissus domestiques. En 1964, Silva Santiesteban propose une analyse généraliste du fonctionnement des \textit{obrajes}\footcite{silvasantistebanObrajesVirreinatoPeru1964}. Les analyses lui succédant sont souvent plus localisées. Miriam Salas Olivari, dans les trois tomes de \textit{Estructura colonial del poder español en el Perú : Huamanga (Ayacucho) a través de sus obrajes. Siglos XVI-XVIII}, propose ainsi une analyse complète du système des \textit{obrajes} dans la région d'Ayacucho (Pérou)\footcite{salasolivariEstructuraColonialPoder1998}. Javier Ortiz de la Tabla Ducasse se concentre sur les \textit{obrajes} de Quito (Équateur) \footcite{ortizdelatabladucasseObrajesObrajerosQuito1982} et Mary Money sur ceux de la province de Charcas (Bolivie)\footcite{moneyObrajesTrajeComercio1983}. Ces différents travaux sur les \textit{obrajes} restent limités et ne détaillent pas forcément les techniques textiles, au profit de l'inscription politique et sociale de ces manufactures. En outre, leur histoire est reconstruite à partir des archives et donc principalement par les témoignages des élites, ce qui ne permet pas de saisir l'influence de ces nouvelles pratiques textiles sur les populations locales. Les Indépendances marquent la fin des \textit{obrajes}, et la production textile cesse d'intéresser les historien\inclusives{ne}, alors même qu'elle perdure dans les communautés de manière \og traditionnelle\fg \:et industriellement pour les vêtements de ville. 

Les \textit{obrajes} sont un élément central de la compréhension des modifications des pratiques textiles à la période coloniale. Toutefois, cette dernière avait tendance à être perçue comme un moment de rupture pour la zone andine, y compris du point de vue des pratiques textiles. Or, depuis les années 1990, des travaux s'attellent à nuancer l'influence européenne, notamment à travers la compréhension des phénomènes de métissage. En effet, à partir de la colonisation du continent, les pratiques textiles ne peuvent être pensées hors de l'histoire des circulations transatlantiques et transpacifiques. Les travaux d'Elena Phipps montrent en effet l'importance des échanges entre la métropole et ses colonies, à la fois en pièces textiles et en savoirs textiles\footcite{phippsIberianGlobe2013}. Ces échanges sont aussi liés à l'arrivée d'artisans européens, notamment de brodeurs, qui importent leurs techniques\footcite[p.~36]{phippsIberianGlobe2013}. Susan Niles se concentre sur les échanges transpacifiques et, à travers l'analyse de pièces coloniales, souligne l'utilisation de nouveaux matériaux et d'une iconographie importés d'Asie\footcite[(p.~60)]{nilesArtistEmpireInca1994}. Toutefois, ces pièces coloniales conservent des traits techniques et iconographiques pré-hispaniques. Maya Stanfield-Mazzi présente en effet certaines pièces coloniales qui auraient été réalisées à partir de modèles européens mais avec des techniques précolombiennes\footcite{stanfield-mazziCreatividadArteAndino2020}. Certaines autrices se sont également intéressées au statut social associé au textile après la Conquête. Elles soulignent le rôle des élites dans les évolutions textiles coloniales\footcite{solorzanogonzalesTapizAndinoNobleza2020} et relèvent que le textile reste un marqueur social hiérarchique dans cette société\footcite{ramosTejidosSociedadColonial2010}. Cependant, ces travaux restent peu nombreux et sont principalement monographiques. La prise en compte de la tradition textile andine d'un point de vue élargi permettrait-elle de mieux saisir les évolutions coloniales, entre concurrence, cohabitation et métissage ?


\subsubsection*{Textiles et Digital}

Comme nous l'avons vu, une grande partie de l'étude des textiles andins a porté sur des questions de classification. L'introduction du numérique dans l'étude du textile a donc, assez naturellement, traité cette problématique, en proposant un ensemble d'ontologies et de base de données. Ces propositions correspondent aussi à une volonté de la recherche universitaire de donner accès aux sources par le numérique.\\

De nombreux musées proposent aujourd'hui l'accès à leurs collections en ligne, tendance qui s'est d'ailleurs renforcée avec la pandémie de \textsc{covid-}\small 19\normalsize. Dans le cadre des projets \textit{Google Arts \& Culture} qui visent à donner accès à des collections stockées dans différents musées, le musée Amano --- musée des textiles précolombiens à Lima --- a rendu accessibles des photographies d'une partie de sa collection. Toutefois, les images ne sont pas accompagnées de métadonnées précises ce qui rend toute investigation scientifique poussée impossible à partir de celles-ci\footcite{AmanoPreColumbianTextile}. Certains musées proposent aussi leurs propres bases de données accessibles et complètes mais qui ne contiennent pas de textiles des Andes\footnote{La base de donnée IMATEX du Centre de document et Musée Textile de Terrassa (province de Barcelone) est particulièrement bien construite. \cite{Imatex}. La bibliothèque de Leeds (Angleterre) met aussi à disposition une collection très complète : \cite{InternationalTextileCollection}.}. Une seule base de données disponible en ligne porte uniquement sur le textile andin. Créée dans le cadre du projet universitaire \og Weaving Communities of Practice \fg\footcite{FrontPageWeaving}, elle contient près de 700 pièces textiles archéologiques et contemporaines produites dans les Andes. 
Denise Y. Arnold, chercheuse principale du projet, a développé la connaissance des pièces textiles andines à travers une approche linguistique des textiles et de leur fabrication\footcite{arnoldHaciaTerminologiaAndina2011}. À partir de l'hypothèse que l'analyse de la langue permet de comprendre les logiques de tissage des femmes, elle aboutit, avec Elvira Espejo, à une proposition de classification des textiles andins reposant sur les termes endogènes mobilisés par les tisserand\inclusives{e} quechuaphones ou aymaraphones\footnote{\cite{arnoldCienciaTejerAndes2019} \\ Le quechua et l'aymara sont les deux langues amérindiennes les plus parlées dans les Andes. À elles deux, elles comptent plus de 10 millions de locuteurs répartis entre le Pérou, la Bolivie, l'Équateur, le Chili et l'Argentine.}. L'approche linguistique est également présentée par ces deux chercheuses comme un outil pour les tisserand\inclusives{e} contemporain\inclusives{e} pour \og offrir ici des informations auparavant disponibles pour les spécialistes et les rendre plus accessibles, dans l'espoir de contribuer au sauvetage et à la revalorisation d'une partie de la complexité technique et technologique antérieure\footnote{\cite[p.~8]{arnoldCienciaTejerAndes2019} Citation originale : \textquotedblleft \textit{ofrecer aquí una información previamente disponible a los especialistas y hacerla más accesible, con la esperanza de contribuir a que se rescate y se revalorice una parte de la complejidad técnica y tecnológica anterior}'\textquotedblright}.\fg \:Denise Y. Arnold et Elvira Espejo font partie de l'équipe de recherche, elles ont notamment encadré une partie de la création technique de la base de données qui s'est faite conjointement entre informaticien\inclusives{ne} et spécialistes du domaine. Leur influence s'observe dans l'attention précise portée aux questions linguistiques, notamment au moment de la création de l'ontologie, comme l'expliquent Richard Brownlow et \textit{al.} dans leur article \og An Ontological Approach to Creating an Andean Weaving Knowledge Base \fg\footcite{brownlowOntologicalApproachCreating2015}.\\

Le second volet de la collaboration entre textile et numérique porte sur l'exploitation des images ou l'expérimentation à partir des techniques textiles. En effet, la rencontre entre techniques textiles et techniques informatiques s'avère fructueuse.

Une grande partie de l'analyse computationnelle du textile vise à des débouchés industriels. La plupart des travaux cherchent des méthodes pour améliorer la production textile contemporaine, notamment les contrôles qualité de l'industrie textile\footcite[p.~6346]{mengAutomaticRecognitionWoven2022}. C'est un domaine qui naît avec l'apparition des ordinateurs ainsi que l'automatisation de la production textile et, dès ses débuts, il vise à une détection des motifs et des techniques utilisées à partir de photographies\footcite{kangAutomaticRecognitionFabric1999}. Le champ se complexifie avec l'introduction de théories mathématiques pour analyser les images de textiles comme, par exemple, la théorie des groupes de symétries\footcite{valienteStructuralDescriptionTextile2004}, la théorie des n\oe{}uds\footnote{\cite{grishanovTopologicalStudyTextile2009} ou encore \cite{helmerSimilarityMeasureWeaving2018}} ou la transformée de Fourier\footnote{\cite{zhengAccurateIndexingClassification2009} ou encore \cite{zhangWeavePatternRecognition2017}}. Cependant, les articles cités précédemment considèrent le tissage comme  \og deux ensembles de fils orthogonaux : une trame passant verticalement et une chaîne passant horizontalement \footnote{\cite[p.~6345]{mengAutomaticRecognitionWoven2022}. Citation originale :  \textquotedblleft two sets of orthogonal yarns: vertically passing warps and horizontally passing wefts\textquotedblright}~\fg. Or, les tissages andins sont souvent composés de plus d'un ensemble de trame et/ou de chaîne et d'autres fils peuvent être adjoints. Par ailleurs, les autres techniques textiles utilisées, comme la broderie, ne sont pas prises en compte dans ces travaux. 

Le développement du \textit{machine learning} et du \textit{deep learning} au sein des analyses computationnelles permet des évolutions dans l'analyse des textiles. Ainsi, des réseaux classificatoires sont utilisés pour calculer la similarité des motifs textiles\footcite{xiangFabricImageRetrieval2019}. C'est notamment une des propositions du projet européen de recherche \textsc{SILKNOW} qui concentre son étude sur la soie européenne du \siecle{xv} au \siecle{xix}\footcite{SILKNOWSILK}. Il repose sur la création d'une base de données ainsi que d'un ensemble d'outils d'analyse des données disponibles, notamment de cartographie digitale. En 2020, une équipe rattachée à ce projet, menée par Franz Rottensteiner, propose d'utiliser des réseaux de neurones convolutifs pour classifier les textiles. Leur but est de développer un algorithme qui soit capable de prédire la date et le lieu de production, la technique et les matériaux utilisés à partir de l'image d'un textile et de ses métadonnées\footcite{clermontAssessingSemanticSimilarity2020}. Dans le cadre de ce même projet, une seconde équipe s'intéresse à la visualisation spatio-temporelle des données des soieries et aboutit à STMaps, \og un outil de visualisation de données spatio-temporelle fondé sur une ontologie\fg\footcite{sevillaMultiPurposeOntologyBasedVisualization2021}. Enfin, le projet \textsc{SILKNOW} propose un simulateur de textile \textit{Virtual Loom} permettant à l'utilisateur de simuler différentes techniques textiles à partir d'une image de son choix. Dans le cas du textile andin, Julian Rohrhuber et David Griffiths sont les seuls à proposer des méthodes computationnelles appliquées aux \textit{quipus}, ces ensembles de cordes nouées utilisés pour stocker des informations numériques dans l'empire Inca\footcite{rohrhuberCodingKnots2017}. Nous pouvons alors nous demander s'il est possible de développer des méthodes similaires à celles de l'équipe de SILKNOW mais adaptée aux tissages andins afin de prédire certaines caractéristiques des textiles.\\

Comme nous l'avons vu, le textile, notamment andin, a été largement étudié par différents champs académiques. Ce travail se positionne aux croisement de ceux-ci, s'attelant à traiter les données importantes fournies par l'archéologie, l'histoire et l'anthropologie grâce aux méthodes numériques contemporaines.

\section*{Problématisation}

%%% Se questionner sur la réalité de la continuité, aujourd'hui défendue par de nombreux artisan.ne.s textiles andins

Cette recherche porte sur les textiles archéologiques et ethnographiques andins en tant qu'objets dynamiques et indicateurs des évolutions des motifs et des techniques. L'inégale répartition des textiles sur le territoire andin est un des principaux écueils à cette compréhension des circulations textiles. Pour pallier ce manque d'informations, nous allons essayer de saisir les \og phénomènes de diffusion, d'innovation et de blocage \fg \:techniques entre les différentes régions du Pérou\footcite[p.~264]{desrosiersTechniquesTissageOntelles2010}. 

Peut-on à partir d'un nombre important d'artefacts saisir la circulation des textiles de cette région, ainsi que les influences de leurs techniques et de leurs iconographies ? Les images nous permettent-elles de détecter les armures textiles, indices des échanges entre civilisations ? Les textiles archéologiques sont-ils un ensemble immuable comme le présente Ciriaco ? Les textiles coloniaux apparaîtront-t-ils comme marqueurs d'un moment de rupture ou bien plutôt dans la continuité des textiles préhispaniques ? \textit{Quid} des textiles ethnographiques, leurs éléments iconographiques et techniques évoluent-ils ?


\section*{Méthodologie}


Comme nous l'avons vu, les musées dans les Andes, mais aussi de nombreux musées en Amérique du Nord et en Europe regorgent de collections de textiles pré-hispaniques andins et la production contemporaine perdure. La zone andine fournit donc une multitude d'exemples de techniques textiles : tissage, tapisserie, broderie etc. Par ailleurs, les textiles ont été largement étudiés en anthropologie et en archéologie mais majoritairement d'un point de vue monographique. Cette configuration des données et des savoirs semble être un terreau bénéfique aux analyses computationnelles. Ces dernières requièrent en effet un nombre important de données pour pouvoir offrir un point de vue global. À partir d'une base de données, déjà constituée, il est alors possible de procéder à différentes analyses et notamment de tenter de pallier l'absence de données archéologiques pour les hautes-terres. 

Pour répondre aux interrogations précédentes, nous proposons de comparer deux indicateurs de circulation des textiles. Les localisations d'une partie des textiles de la base de données sont connues et permettraient donc de déterminer des zones de productions textiles importantes ainsi que les distances qui séparent les textiles, ceci dans une approche diachronique. Ces informations géographiques seront comparées aux distances iconographiques et techniques des textiles, calculées au moyen de réseaux de neurones pré-entraînés sur des images de textiles. Ces distances iconographiques et techniques, croisées à la géographie des textiles, permettront d'obtenir des indicateurs d'influence entre différents foyers textiles.



\section*{Annonce de plan}
Pour mener cette étude, nous présenterons d'abord le cadre théorique des circulations textiles. Nous commencerons par aborder le concept de \og l'andin \fg \:pour essayer de comprendre la catégorie textile sur laquelle se fonde notre analyse. Puis, nous nous questionnerons sur les indices attestant des échanges textiles et des influences techniques et iconographiques dans les Andes. 

Dans une seconde partie, nous explorerons les données présentes dans la base de données en nous concentrant sur les données géographiques associées aux textiles. Une phase de cartographie visera à visualiser les foyers textiles et les échanges ayant lieu entre ces foyers.

Enfin, nous approfondirons la compréhension des échanges textiles en ayant recours à des outils de \textit{computer vision}. Nous mènerons deux classifications des textiles, supervisée et non-supervisée, pour essayer de saisir les influences et ré-interprétations techniques et iconographiques.

